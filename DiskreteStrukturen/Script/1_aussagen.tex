%%
%% Author: Abgie
%% 16.10.2018
%%

% Preamble
\documentclass[diskretestrukturen.tex]{subfiles}

% Document
\begin{document}
  \section{Aussagen}
  \begin{definition}[Aussage]
    Eine Aussage ist eine Repr\"asentation eines Satzes, der entweder wahr (1) oder falsdch (0) ist.
  \end{definition}

  \begin{description}
    \begin{itemize}
      \item nich die Wahrheitsbestimmung von Basis-Aussagen
      \item Formalisierung von Aussagenverk\"upfungen
      \item Bewertung von Aussagenverk\"upfungen basierend auf Wahrheitswerden der Teilaussagen
      \item Schlussregeln
    \end{itemize}
    \item \textbf{Notation (Junktoren):}
    \begin{itemize}
      \item Negation: $\neg A$ (nicht $A$)
      \item Konjunktion: $A \bigwedge B$ ($A$ und $B$)
      \item Disjunktion: $A \bigvee B$ ($A$ oder $B$)
      \item Implikation:  $A \rightarrow B$ (wenn $A$, dann $B$)
      \item ACHTUNG: Implikation ist genau dann falsch, wenn $A$ gilt und die Folgerung $B$ nicht.
    \end{itemize}
    \item \textbf{Syntax und Semantik}
    \begin{definition}[Atome und Formeln]
      Aussagenlogische Atome sind primitive Aussagen wie $A$ oder $B$.

      Aussagenlogische Formeln sind Aussagen inklusive Verkn\"upfungen.


      W\"ahrend die Wahrheit eines Atoms abh\"angig von der fachlichen Aussage ist, ist die Wahrheit einer Formel nur von der Wahrheit
      ihrer Atome abh\"angig.

      Als Beweismethode wird die Wahrheitswertetabelle genutzt.
    \end{definition}
    \item \textbf{Wahrheitswertetabelle:}
    \begin{itemize}
      \item Beweisschema f\"ur komplexe Aussagen
      \item tabellarische Auflistung aller M\"oglichkeiten
    \end{itemize}
    \item \textbf{\"Aquivalenz:}
    \begin{definition}[\"Aquivalenz]
      Zwei Aussagen $A$ und $B$ sind \"aquivalent ($A \leftrightarrow B$), genau dann wenn deren Wahrheitswerte uebereinstimmen.
    \end{definition}


    \"Aquivalente Formeln:
    \begin{tabular}{r|rr}
      \hline
      \"aquivalente Formeln & \"aquivalente Formeln && Bezeichnung  \\
      \hline
      $A \bigwedge B$ && $B \bigwedge A$ && Kommutativit\"at von $\bigwedge$  \\
      $A \bigvee B$ && $B \bigvee A$ && Kommutativit\"at von $\bigvee$  \\

      $(A \bigwedge B) \bigwedge C$ && $A \bigwedge (B \bigwedge C)$ && Assoziativit\"at von $\bigwedge$  \\
      $(A \bigvee B) \bigvee C$ && $A \bigvee (B \bigvee C)$ && Assoziativit\"at von $\bigvee$  \\

      $A \bigwedge (B \bigvee C)$ && $(A \bigwedge B) \bigvee (A \bigwedge C)$ && Assoziativit\"at von $\bigwedge$  \\
      $A \bigvee (B \bigwedge C)$ && $(A \bigvee B) \bigwedge (A \bigvee C)$ && Assoziativit\"at von $\bigvee$  \\

      $A \bigwedge A$ && $A$ && Idempotenz von $\bigwedge$  \\
      $A \bigvee A$ && $A$ && Idempotenz von $\bigvee$  \\

      $\neg\neg A$ && $A$ && Involution von $\neg$  \\

      $\neg(A \bigwedge B)$ && $(\neg A) \bigvee (\neg B)$ && DeMorgan-Gesetz f\"ur $\bigwedge$  \\
      $\neg(A \bigvee B)$ && $(\neg A) \bigwedge (\neg B)$ && DeMorgan-Gesetz f\"ur $\bigvee$  \\
      \hline
    \end{tabular}
  \end{description}

\end{document}