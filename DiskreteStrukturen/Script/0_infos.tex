%%
%% Author: Abgie
%% 16.10.2018
%%

% Preamble
\documentclass[diskretestrukturen.tex]{subfiles}

% Document
\begin{document}
  \begin{itemize}
    \item Termin: dienstags, 17:15 - 18:45 Uhr, Audimax
    \item Beginn: \textbf{16.10.2018}
  \end{itemize}

  Die Einschreibung ist bis sp\"atestens 22.10. in Almaweb vorzunehmen.

  \section*{\"Ubersicht}
  \textbf{F\"ahigkeiten:}

  \begin{itemize}
    \item Standardnotizen lesen und schreiben
    \item Einf\"uhrung mathematisches Denken
    \item Beweise lesen und analysieren
    \item formale Beweise f\"uhren
  \end{itemize}

  \begin{itemize}
    \item Erwartete Vorkenntnisse: Informatik-Grundkenntnisse
  \end{itemize}


  \section*{Gliederung}
  \begin{enumerate}
    \item Aussagen- und Pr\"adikatenlogik
    \item Naive Mengenlehre
    \item Relationen und Funktionen
    \item Kombinatorik und Stochastik
    \item Algebraische Strukturen
    \item B\"aume und Graphen
    \item Arithmetik
  \end{enumerate}


  \section*{Leistungsbewertung}
  Es werden 5 Leistungspunkte f\"ur DS vergeben.
  \begin{itemize}
    \item Pr\"ufungsvorleistung: Zwischenklausur (60 Minuten)
    \item keine m\"undliche Pr\"ufung
  \end{itemize}

  Der Klausurerfolg erfordert:
  \begin{itemize}
    \item Wissen \"uber die Vorlesungsinhalte
    \item Kenntnisse und Fertigkeiten zur Anwendung des Wissens
    \item Vorlesungsteilnahme, Vorlesungsnachbearbeitung
    \item mind. 50\% der Aufgaben
    \"Ubungen
  \end{itemize}


  \section*{\"Ubungsbetrieb}
  \begin{description}
    \item \textbf{\"Ubungsbl\"atter:}
    \begin{itemize}
      \item ca. alle 2 Wochen auf \url{almaweb.uni-leipzig.de} (ab 22.10.)
      \item Besprechung jeweils ab \textbf{1 Woche sp\"ater} in den \"Ubungen
      \item Abgabe der Aufgaben vor der Vorlesung
      \item \textbf{\"Ubungsleiter:} Erik Paul
      \item Bearbeitung erforderlich mit Bewertung
    \end{itemize}
  \end{description}

  Im Folgenden die vorl\"aufige Angabe der einzelnen \"Ubungstermine pro Gruppe:

  \begin{tabular}{r|rrrrrrr}
    Blatt & Blatt 1 & Blatt 2 & Blatt 3 & Blatt 4 & Blatt 5 & Blatt 6 & Blatt 7  \\
    \hline
    Abgabe & 30.10. & 13.11. & 27.11. & 11.12. & 08.01. & 22.01. & 05.02.  \\
    Seminar (\"UB01e) & 25.10. & 08.11. & 22.11. & 06.12. & 03.01. & 17.01. & 31.01.  \\
  \end{tabular}

  Es sind Benuspunkte m\"oglich:

  \begin{tabular}{r|r}
    \hline
    Punkte (in \%) & Konzequenz  \\
    \hline
    $<$ 50 & Pr\"ufungsteilnahme \"uberdenken  \\
    50--59 & Pr\"ufung vermutlich machbar  \\
    60--74 & +1 Bonuspunkt (3\%) f\"ur die Pr\"ufung  \\
    75--89 & +2 Bonuspunkte (6\%) f\"ur die Pr\"ufung  \\
    $>$ 89 & +3 Bonuspunkte (10\%) f\"ur die Pr\"ufung  \\
    \hline
  \end{tabular}

  Folien und Ank\"undigungen gibt es im OLAT-Kurs: W14.Inf.DiskreteStrukturen

\end{document}